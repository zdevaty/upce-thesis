\documentclass[a4paper,12pt,twoside]{article}
% tento dokument používá balíky specifické pro XeLaTeX a lze jej přeložit
% jen XeLaTeXem, nemáte-li instalována použitá (komerční) písma Myriad Pro
% a Vida Mono 32 Pro, vymažte volbu fonts u balíku tul a příkaz \setmonofont

\newcommand{\verze}{2.0}

\usepackage{polyglossia}
\setdefaultlanguage{czech}

\usepackage{fontspec}
\usepackage{xunicode}

\usepackage{xltxtra}
\usepackage{tabularx}

\usepackage{makeidx}
\makeindex

% živé odkazy v PDF
\usepackage{hyperref}
\hypersetup{colorlinks=true, linkcolor=tul, urlcolor=tul, citecolor=tul}
\hypersetup{pdftitle={Návod k použití balíku tul pro LaTeX verze \verze}}

\usepackage[fonts,noheader]{tul}
\TULname{Pavel Satrapa}
%\TULposition{prorektor pro informatiku}
\TULphone{+420\,485\,351\,234}
\TULmail{Pavel.Satrapa@tul.cz}

% definice příkazů a prostředí specifických pro tento dokument
\newcommand{\cmdfont}[1]{\texttt{\color{\tulcolor}#1}}
\newcommand{\cmdnoindex}[1]{\cmdfont{\textbackslash #1}}
\newcommand{\cmd}[1]{\cmdnoindex{#1}\index{#1@\textbackslash #1}}
\newcommand{\demobox}{\raisebox{-.20ex}{\rule{1em}{1em}}}
\newenvironment{itemize*}%
  {\begin{itemize}%
    \setlength{\parskip}{0.25\baselineskip}%
    \setlength{\itemsep}{0pt}%
    \setlength{\partopsep}{0pt}%
    \setlength{\topsep}{0pt}}%
  {\end{itemize}}


\parindent=0pt
\parskip=0.5\baselineskip
\sloppy

\newcommand{\cmddemo}[1]{\bigskip\parbox[c]{3.9cm}{\cmd{#1}}\parbox[c]{10cm}{\csname #1\endcsname}\bigskip}

\begin{document}

\TULfancytitlepage{Návod k~použití balíku tul pro \LaTeX\ (verze~\verze)}
{Pavel Satrapa}
{Liberec 2022}


Cílem balíku \cmdfont{tul} je usnadnit vytváření dokumentů odpovídajících
vizuálnímu stylu Technické univerzity v~Liberci a jejích součástí
v~typografickém programu \LaTeX. Balík definuje:

\begin{itemize*}
\item univerzitní barvy,
\item grafické prvky (loga, symbol),
\item záhlaví a zápatí stránek,
\item vzhled nadpisů pro strukturování textu,
\item písma (pokud jsou v~systému k~dispozici).
\end{itemize*}


\section{Předpoklady}

Balík \cmdfont{tul} ke své činnosti potřebuje několik dalších balíků. Jedná
se o~běžně dostupné balíky, které bývají obvyklou součástí instalací (jako je
například \TeX\ Live):

\begin{itemize*}
\item fancyhdr,
\item graphicx,
\item ifthen,
\item titlesec,
\item xcolor.
\end{itemize*}

Jelikož loga jsou reprezentována obrázky ve formátu PDF, je třeba, aby použitá
implementace \TeX u podporovala tento grafický formát. Ideální je implementace,
která přímo vytváří PDF výstup, jako je například pdf\LaTeX\ nebo \XeLaTeX.


\section{Použití}

Balík je tvořen několika soubory: základ tvoří soubor \cmdfont{tul.sty}
obsahující vlastní příkazy. Doprovází jej jednotlivé grafické prvky
s~příslušnými logy~-- soubory \cmdfont{logo-*.pdf} a \cmdfont{symbol-*.pdf}.
Balík doplňuje soubor \cmdfont{tulthesis.cls} definující třídu pro sazbu
absolventských prací TUL. Umístěte tyto soubory do vhodné složky, kde je vaše
instalace \LaTeX u najde. Doporučujeme zřídit pro balík samostatný adresář ve
složce, kde se nacházejí ostatní balíky vaší instalace. Nezapomeňte
aktualizovat databázi souborů, aby \LaTeX\ soubory skutečně našel (příkazem
\cmdfont{texhash}, \cmdfont{mktexlsr} nebo jeho ekvivalentem).

Vlastní použití v~dokumentech je zcela standardní~-- v~preambuli uveďte:

\begin{quote}
\cmdnoindex{usepackage\{tul\}}
\end{quote}


\subsection*{Volby balíku}

Základní sada voleb slouží k~přepínání mezi vizuálními styly jednotlivých
fakult a ústavů TUL. Pokud nepoužijete žádnou z~nich, bude dokument sázen ve
vizuálním stylu celé univerzity. Jsou k~dispozici tyto \textbf{volby součástí
TUL}: \cmdfont{FS}, \cmdfont{FT}, \cmdfont{FP}, \cmdfont{EF}, \cmdfont{FUA},
\cmdfont{FM}, \cmdfont{FZS}, \cmdfont{CXI}. Použití některé z~nich změní
výchozí barvu, logo a styl zápatí stránek. Používejte vždy nanejvýš jednu
z~těchto voleb.

Kromě přepínání součástí TUL jsou k~dispozici následující volby:

\medskip

\begin{tabularx}{\textwidth}{@{}lX}
\cmdfont{EN} & přepne na anglickou verzi~-- týká se generických příkazů
(např. \cmd{logo}) a nápisů v~zápatí,\\
\cmdfont{bw} & sází se černobíle~-- změní loga na černobílá, nadpisy největších
částí textu budou černé a univerzitní či fakultní barva je nahrazena šedou,\\
\cmdfont{bwtitles} & nadpisy největších částí textu budou černé, jinde barva
zůstane,\\
\cmdfont{noheader} & neumisťovat logo do záhlaví stránek,\\
\cmdfont{fonts} & text dokumentu má být sázen písmem podle vizuálního stylu,
více v části~\ref{pisma} na straně~\pageref{pisma}.
\end{tabularx}

\medskip

Například dokument ve stylu Fakulty mechatroniky bude v~preambuli obsahovat:

\begin{quote}
\cmdnoindex{usepackage[FM]\{tul\}}
\end{quote}

Jestliže je navíc cílem černobílý dokument, změní se použití balíku na:

\begin{quote}
\cmdnoindex{usepackage[FM,bw]\{tul\}}
\end{quote}


\section{Barvy}

Balík definuje sadu barev pro TUL a její jednotlivé fakulty, které odpovídají
vizuálnímu stylu TUL. Lze je používat v~příkazech \cmdnoindex{color} a podobných,
stejně jako standardní předdefinované barvy. Jsou k~dispozici následující
\textbf{názvy barev}:

\begin{center}
\begin{tabular}{l@{\qquad}l@{\qquad}l@{\qquad}l@{\qquad}l}
{\color{tul}\demobox}\ \cmdfont{tul} &
{\color{tulFS}\demobox}\ \cmdfont{tulFS} &
{\color{tulFT}\demobox}\ \cmdfont{tulFT} &
{\color{tulFP}\demobox}\ \cmdfont{tulFP} \\
{\color{tulEF}\demobox}\ \cmdfont{tulEF} &
{\color{tulFUA}\demobox}\ \cmdfont{tulFUA} &
{\color{tulFM}\demobox}\ \cmdfont{tulFM} &
{\color{tulFZS}\demobox}\ \cmdfont{tulFZS} &
{\color{tulCXI}\demobox}\ \cmdfont{tulCXI}
\end{tabular}
\end{center}

Kromě těchto pevně daných názvů barev je k~dispozici i příkaz \cmd{tulcolor},
který obsahuje název barvy té součásti univerzity, v~jejímž stylu je dokument
sázen. Hodnota \cmd{tulcolor} závisí na „fakultní“ volbě při použití
balíku. Například barva nadpisů v~tomto dokumentu byla vytvořena pomocí
\cmdnoindex{color\{\textbackslash tulcolor\}}.


\section{Grafické prvky}

Sada příkazů umožňuje vkládat do textu \textbf{loga} a podobné grafické prvky. Pro
úpravu jejich velikosti použijte příkazy \cmdnoindex{scalebox} nebo
\cmdnoindex{resizebox}. Loga neobsahují ochrannou zónu, příslušné volné místo
kolem nich musí zajistit autor dokumentu.

Pro obecné konstrukce je k~dispozici několik příkazů, jejichž funkce závisí na
volbách balíku. Vysází aktuální krátké či dlouhé logo a symbol TUL nebo té její
součásti, kterou jste vybrali. Pokud jste například balík vložili s~volbami
\cmdfont{FS} a \cmdfont{EN}, příkaz \cmd{logo} vysází anglickou verzi loga
Fakulty strojní. Jedná se o tyto příkazy:

\begin{itemize*}
\item \cmd{logo} -- základní (krátké) logo,
\item \cmd{logotext} -- plná textová verze loga,
\item \cmd{CZlogo}, \cmd{CZlogotext}, \cmd{ENlogo}, \cmd{ENlogotext} --
analogie předchozích, ale logo je vždy ve zvolené jazykové verzi, bez ohlednu
na volby balíku,
\item \cmd{symbolTUL} -- symbol v barvě součásti.
\end{itemize*}

Jejich aktuální podoba vypadá takto:

\cmddemo{logo}

\cmddemo{logotext}

\cmddemo{CZlogo}

\cmddemo{CZlogotext}

\cmddemo{ENlogo}

\cmddemo{ENlogotext}

\cmddemo{symbolTUL}

Příbuzný je příkaz \cmd{TULsoucast}, který vloží název aktuální součásti
univerzity v~aktuální jazykové verzi. Při volbách \cmdfont{FS} a \cmdfont{EN}
tedy vysází „Faculty of Mechanical Engineering“. Pomocí \cmd{CZTULsoucast} a
\cmd{ENTULsoucast} lze vysázet název aktuální části v~požadovaném jazyce.
Nejedná se o~grafické prvky, ale o~běžná textová makra. Do této kapitoly jsem
je zařadil vzhledem k~příbuznosti s~předchozími příkazy.

Následující příkazy sází konkrétní loga TUL a jednotlivých součástí:

\cmddemo{logoTUL}

\cmddemo{logoTULtext}

\cmddemo{logoTULalt}

\cmddemo{logoFS}

\cmddemo{logoFStext}

\cmddemo{logoFT}

\cmddemo{logoFTtext}

\cmddemo{logoFP}

\cmddemo{logoFPtext}

\cmddemo{logoEF}

\cmddemo{logoEFtext}

\cmddemo{logoFUA}

\cmddemo{logoFUAtext}

\cmddemo{logoFM}

\cmddemo{logoFMtext}

\cmddemo{logoCXI}

\cmddemo{logoCXItext}

\cmddemo{logoFZS}

\cmddemo{logoFZStext}


Symbol TUL je jako jediný grafický prvek přístupný v~černobílé variantě, i pokud se
sází barevně. Slouží k~tomu příkaz \cmd{symbolTULbw}. Loga univerzity,
jednotlivých fakult a ústavů jsou k~dispozici pouze v~jedné variantě~-- bez
volby \cmdfont{bw} se jedná o~verzi barevnou, pokud ale použijete balík
s~volbou \cmdfont{bw}, jsou všechna loga (včetně \cmd{symbolTUL}) černobílá.

\cmddemo{symbolTULbw}

Veškerá loga jsou k dispozici i v anglické verzi, jednotlivé příkazy mají
předponu \cmdfont{EN} (a to i v případě, že se anglické logo neliší od českého).

\cmddemo{ENlogoTUL}

\cmddemo{ENlogoTULtext}

\cmddemo{ENlogoTULalt}

\cmddemo{ENlogoFS}

\cmddemo{ENlogoFStext}

\cmddemo{ENlogoFT}

\cmddemo{ENlogoFTtext}

\cmddemo{ENlogoFP}

\cmddemo{ENlogoFPtext}

\cmddemo{ENlogoEF}

\cmddemo{ENlogoEFtext}

\cmddemo{ENlogoFUA}

\cmddemo{ENlogoFUAtext}

\cmddemo{ENlogoFM}

\cmddemo{ENlogoFMtext}

\cmddemo{ENlogoFZS}

\cmddemo{ENlogoFZStext}

\cmddemo{ENlogoCXI}

\cmddemo{ENlogoCXItext}

Pro použití na tmavém pozadí jsou dále k dispozici všechna loga v bílém provedení.
Názvy jejich příkazů mají příponu \cmdfont{wh}~-- například
\cmdnoindex{logowh}, \cmdnoindex{logoFPwh}, \cmdnoindex{ENlogoFStextwh} atd.

Je také definován příkaz \cmd{darkTULbg}, který změní příkazy \cmd{logo},
\cmd{logotext}, \cmd{symbolTUL} a \cmd{logoTULalt} tak, aby sázely bílé verze
příslušných log.


\section{Záhlaví a zápatí stránky}

Použití stylu automaticky nastaví styl stránky \cmdfont{TUL}, který v~levém
horním rohu nese logo, vpravo symbol, ve spodní části jsou kontaktní informace,
na vnějším okraji pak číslo stránky.

Tento formát je vhodný pro krátké dokumenty (např. dopisy). V~případě delších
textů může přemíra prvků „dekorujících“ stránku škodit. Proto je k~dispozici
několik alternativ, jimiž lze doprovodné prvky stránek měnit.

Způsob úpravy záhlaví a zápatí byl proti verzi~1 upraven, nyní balík definuje
několik stránkových stylů, které lze aktivovat standardními příkazy
\cmdfont{\textbackslash pagestyle} a \cmdfont{\textbackslash thispagestyle}.
Dostupné možnosti shrnuje tabulka~\ref{styly-stranek}.

\begin{table}[htp]
\begin{center}
\newcommand{\ano}{\mbox{}\qquad ✓\qquad\mbox{}}
\begin{tabular}{l@{}c@{}c@{}c}
\emph{styl} & \emph{logo} & \emph{zápatí} & \emph{číslo str.} \\
\cmdfont{TUL}             & \ano & \ano & \ano \\
\cmdfont{TULnopage}       & \ano & \ano &   \\
\cmdfont{TULheader}       & \ano &   & \ano \\
\cmdfont{TULheaderonly}   & \ano &   &   \\
\cmdfont{TULfooter}       &   & \ano & \ano \\
\cmdfont{TULfooternopage} &   & \ano &   \\
\cmdfont{TULfooternoaddr} &   &   & \ano \\
\end{tabular}
\end{center}
\caption{Styly stránek}
\label{styly-stranek}
\end{table}

Liší se přítomností tří prvků: loga a symbolu v záhlaví, kontaktních informací
v zápatí a čísla stránky. Například pro jednostránkový dokument je vhodný styl
\cmdfont{TULnopage}. Naopak pro rozsáhlejší vícestránkové texty kdy chcete
umístit logo jen na titulní stránku, je vhodnější použít volbu balíku
\cmdfont{noheader}, která záhlaví s~logem zcela odstraní. Na začátku titulní
stránky pak můžete použít například příkazy:

\begin{quote}
\cmdnoindex{logo}\\
\cmdnoindex{vspace\{1cm\}}
\end{quote}

nebo sáhnout po připravené titulní stránce, kterou balík také nabízí (viz
dále). Tímto způsobem byl sázen tento dokument.

Pro zachování zpětné kompatibility jsou nadále k dispozici původní příkazy
\cmd{TULheader}, \cmd{noTULheader}, \cmd{TULfooter}, \cmd{TULfooternopage},
\cmd{footaddress} a \cmd{nofootaddress}.

V~adrese je možné \textbf{změnit osobní a kontaktní údaje}, pokud autor
dokumentu chce vložit své vlastní. Slouží k~tomu příkazy:

\begin{itemize*}
\item \cmd{TULname} -- jméno,
\item \cmd{TULposition} -- funkce,
\item \cmd{TULphone} -- telefonní číslo; uvádějte celé, trojice číslic oddělujte
tenkými mezerami~(\cmdnoindex{,}),
\item \cmd{TULmail} -- e-mail.
\end{itemize*}

Jejich argumentem je nová hodnota. Příkazy jsou nepovinné. Pokud je neuvedete,
jméno a funkce se vynechá, za telefonní číslo a e-mail se dosadí obecné
hodnoty. Příkazy fungují jako přepínače~-- od okamžiku jejich použití bude
nadále platit zadaná hodnota. Doporučuji použít je v~preambuli a v~těle
dokumentu již údaje neměnit. Příklad použití:

\begin{quote}
\cmdnoindex{TULname\{Pavel Satrapa\}}\\
\cmdnoindex{TULmail\{Pavel.Satrapa@tul.cz\}}\\
\cmdnoindex{TULphone\{+420\textbackslash ,485\textbackslash ,351\textbackslash ,234\}}
\end{quote}

Balík se snaží při výpočtu rozměrů tiskového zrcadla a okrajů přizpůsobit
použitému formátu papíru (volby \cmdfont{a4paper} a další pro třídu dokumentu).
Pro formát~A4 balík nastaví rozměry po svém, aby byla zachována zpětná
kompatibilita s předchozími verzemi. Pokud vám výsledky nevyhovují, je třeba po
použití balíku upravit příslušné parametry.

Záhlaví a zápatí je realizováno balíkem \cmdfont{fancyhdr}. Pro případné zásahy
do podoby těchto částí stránky použijte příkazy tohoto balíku.


\section{Nadpisy částí textu}

Balík automaticky aktivuje sazbu nadpisů všech částí textu tučným bezserifovým
písmem. U~dvou nejvyšších úrovní (\cmdnoindex{section} a \cmdnoindex{chapter},
pokud použitá třída tuto úroveň definuje) zároveň sází nadpis fakultní barvou a
při použití volby \cmdfont{fonts} písmem TUL Mono.

Tyto funkce jsou implementovány pomocí balíku \cmdfont{titlesec}. Pokud je
chcete změnit, použijte jeho rozhraní, konkrétně příkaz
\cmdnoindex{titleformat*}.


\section{Písma}\label{pisma}

Součástí grafické identity TUL je písmo \textbf{TUL Mono}. Jedná se o
neproporcionální písmo, které tvarově vychází z šablon, jimiž se
popisovaly technické výkresy.

{\TULmono Ukázka písma TUL Mono.}

Písmo obsahuje pouze verzálky a celkově je vzhledem ke svému charakteru určeno
pro nadpisy a krátké nápisy. Pro delší texty se nehodí. Existuje ve čtyřech
tučnostech:

\begin{itemize}\setlength{\itemsep}{0pt}
\item Regular~-- základní řez
\item Bold~-- tučné
\item Hairline~-- velmi tenké
\item SingleLine~-- s nulovou tloušťkou, určeno pro vyřezávání
\end{itemize}

\begin{figure}[htp]
\begin{center}
{\TULmono Regular \textbf{Bold} \fontspec{TUL Mono Hairline}Hairline
\fontspec{TUL Mono SingleLine}SingleLine}\\
\end{center}
\caption{Verze písma TUL Mono}
\label{tul-mono}
\end{figure}

Pro běžné texty předepisuje grafická identita TUL bezserifové písmo
\textbf{Inter}, případně serifové \textbf{Merriweather}. Jako jejich doplněk
pro ukázky kódu či konfiguračních souborů neproporcionální písmo \textbf{Noto
Sans Mono}.


\subsection{Instalace písem}

Všechna zmiňovaná písma jsou volně šiřitelná a jsou zahrnuta do distribuce
tohoto balíku. Najdete je v adresáři \cmdfont{pisma}. Jsou ve formátu OpenType,
který podporují jen novější implementace \LaTeX u: \XeLaTeX\ nebo Lua\LaTeX.
Starší pdf\LaTeX\ tento formát písem neumí.

Zmiňované implementace pracují se systémovými písmy. Abyste zde uvedená písma
mohli používat, musíte je nejprve instalovat do svého operačního systému. V
grafických systémech je obvykle nejjednodušší klepnout na soubor písma pravým
tlačítkem myši a z menu vybrat instalaci.

V Linuxu můžete soubory zkopírovat do adresáře se systémovými písmy (obvykle
\cmdfont{/usr/share/fonts/opentype}), ve kterém si pro ně nejlépe vytvořte
podadresář. Následně aktualizujte seznam systémových písem příkazem
\cmdfont{fc-cache}.


\subsection{Použití}

Vzhledem k tomu, že písma vyžadují překlad \XeLaTeX em nebo Lua\LaTeX em, balík
\cmdfont{tul} standardně do písem nijak nezasahuje. Jejich použití je potřeba
aktivovat volbou \cmdfont{fonts} při vložení balíku. Tato volba má následující
účinky:

\begin{itemize*}
\item Upraví příkaz \cmd{TULmono}.
\item Změní standardní rodiny písem následovně:\\
\begin{tabular}{@{}ll}
\cmdfont{\textbackslash rmfamily} & Inter, \\
\cmdfont{\textbackslash sffamily} & Inter, \\
\cmdfont{\textbackslash ttfamily} & Noto Sans Mono.
\end{tabular}
\end{itemize*}

Písmo Merriweather není standardně aktivováno, preferováno je používání
bezserifového písma Inter.

Příkaz \cmd{TULmono} přepne na písmo TUL Mono (bez volby \cmdfont{fonts} přepne
na tučné bezserifové písmo). Chová se podobně jako například standardní
\cmdfont{\textbackslash sffamily}. Na tučný řez se lze přepnout standardními
příkazy pro tučné písmo, např. \cmdfont{\textbackslash textbf}. S využitím
Hairline a SingleLine se v \LaTeX u nepočítá.

Specialitou písma TUL Mono jsou kontextové slitky, kterými lze připojením znaku
„\&“ vytvořit podtržené nápisy TUL a Liberec (jen tyto dva). Ve zdrojovém kódu pište
\cmdfont{TUL\textbackslash \&}, resp. \cmdfont{Liberec\textbackslash \&} nebo
\cmdfont{Liberci\textbackslash \&}.

\begin{center}
{\TULmono Máme rádi Liberec\&.}
\end{center}


\section{Titulní stránka}

Součástí balíku jsou i příkazy pro vysázení titulní strany v~příslušném stylu.
K~dispozici jsou dvě alternativy: \cmd{TULtitlepage} pro umírněnou verzi
s~bílým pozadím a \cmd{TULfancytitlepage} pro verzi s pozadím v barvě
univerzity či její součásti. Až na název příkazu mají oba shodný tvar použití:

\begin{quote}
\cmdnoindex{TULtitlepage\{\emph{název}\}\{\emph{autor}\}\{\emph{datum}\}}
\end{quote}

V~argumentech lze používat jednoduché formátovací příkazy, nicméně
doporučuji omezit se na případné rozdělování do více řádků pomocí
\cmdnoindex{\textbackslash}~-- obvykle k~oddělení několika autorů nebo rozdělení
názvu do více řádků. Například titulní strana tohoto dokumentu vznikla
příkazem

\begin{quote}\begin{flushleft}
\cmdnoindex{TULfancytitlepage\{Návod k~použití balíku tul pro \LaTeX\
(verze\textasciitilde\textbackslash verze)\}\{Pavel Satrapa\}\{Liberec 2022}\}
\end{flushleft}\end{quote}


\section{Příklad}

Tento text je příkladem použití balíku \cmdfont{tul}. Jedná se o~celkem
standardní zdrojový text třídy \cmdfont{article}, jehož jedinou úpravou bylo
použití příkazů

\begin{quote}
\cmdnoindex{usepackage[fonts,noheader]\{tul\}}\\
\cmdnoindex{TULname\{Pavel Satrapa\}}\\
\cmdnoindex{TULmail\{Pavel.Satrapa@tul.cz\}}\\
\cmdnoindex{TULphone\{+420\,485\,351\,234\}}
\end{quote}

v~preambuli.


\renewcommand{\indexname}{Přehled příkazů}
\printindex

\end{document}

